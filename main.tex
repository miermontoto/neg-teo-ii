\documentclass[12pt, twoside, openany]{book}

\usepackage{mathptmx} % Contiene una fuente similar a Times New Roman

\usepackage[spanish, es-tabla]{babel} % Permite escritura en castellano
\usepackage[utf8]{inputenc} % Permite utilizar caracteres UTF8

\usepackage{graphicx} % Para la inclusión de gráficos e imágenes
\graphicspath{ {images/} } % Ruta para buscar las imágenes
\usepackage[a4paper,top=30mm,left=30mm,right=25mm,bottom=25mm,headheight=20mm]{geometry} % Configuración de los margenes de la página

% Paquetes para que funcione el formato.
\usepackage{titlesec}
\usepackage{setspace}
\usepackage{ragged2e}
\usepackage{fancyhdr}
\usepackage{lastpage}
\usepackage{stackengine}
\usepackage{array}
\usepackage{hyperref}
\usepackage{float}

\usepackage{hyperref} % Paquete para que las referencias funcionen, y permite introducir links
\usepackage{xcolor} % Paquete para trabajar con colores (fondo de celdas, color del texto...)
\usepackage{pdfpages} % Paquete para incluir páginas de otro PDF directamente
\input{config.tex}


\begin{document}

\rmfamily % Fuente tipo Romana

% Portada de la memoria
\input{title.tex}

% Índice de contenido
\addcontentsline{toc}{chapter}{Índice de contenido} % Añade la referencia al índice de contenido
\tableofcontents
\newpage

% Índice de figuras
\addcontentsline{toc}{chapter}{Índice de figuras}  % Añade la referencia al índice de contenido
\listoffigures
\newpage

% Índice de tablas
\addcontentsline{toc}{chapter}{Índice de tablas} % Añade la referencia al índice de contenido
\listoftables


\justify{} % Texto justificado
\setlength{\parskip}{\baselineskip} % Separación entre párrafos de 1 linea
\onehalfspacing{} % Interlineado de 1,5

%% El contenido de la memoria, dividido en capítulos:
\chapter{Introducción}
\section{Definiciones}
\subsection{Minería de anomalías}
En esencia, la minería de anomalías (también conocida como \textit{minería de excepciones} o
\textit{detección de valores atípicos}) es una rama de la minería de datos que se enfoca en
la detección de patrones anómalos o inusuales en los datos.

Estas anomalías representan desviaciones significativas de los comportamientos esperados
o de las tendencias habituales en un entorno comercial.
\subsection{Anomalía}
Existen múltiples posibles definiciones de anomalía. Algunas de ellas son:
\begin{itemize}[topsep=0pt]
	\item Una observación que se desvía tanto de las demás como para despertar sospechas de
		que se ha generado por un mecanismo diferente.
	\item Casos o conjuntos de datos que aparecen muy raramente y cuyas características
		difieren significativamente de los demás.
	\item Patrones en los datos que no se ajustan a una noción bien definida de comportamiento
		normal.
\end{itemize}

\section{Objetivos y aplicaciones en el mundo real}
El objetivo principal radica en identificar estos eventos ``atípicos'' que podrían llegar
a indicar problemas o posibles oportunidades, si lo observamos desde el punto de vista de
la \subject.

En el mundo real, la minería de anomalías se puede aplicar (y se aplica) a la hora de
tomar decisiones empresariales, como por ejemplo al detectar fraudes, revelar áreas de
mejora, detectar problemas operativos\ldots

En apartados posteriores se verán ejemplos de aplicaciones y casos de uso de la minería de
excepciones.~(Ver \nameref{chap:cus})

\section{Causas de las anomalías}
Las anomalías pueden ser causadas por una amplia variedad de razones, pero existen algunas
causas comunes que pueden ser identificadas:

\begin{itemize}
	\item \textbf{Datos de clases diferentes:} un objeto puede ser diferente del resto por
		pertencer a una clase diferente. En el ejemplo de los fraudes, los objetos ``outliers''
		dentro de un conjunto de datos pueden ser considerados como inusuales o como potenciales
		fraudes.
	\item \textbf{Variación natural:} cuando los datos se pueden modelar estadísticamente
		mediante una distribución, los objetos que se encuentran en las colas de la distribución
		se pueden llegar a considerar como anomalías.
	\item \textbf{Errores en la medición:} todos los datos que se recogen están sometidos a una
		componente humana y, por lo tanto, a errores. Dichos errores pueden llegar a ser
		considerados anomalías, por lo que hay que lidiar con ellos en la fase de preprocesamiento.
		(Ver \nameref{sec:pre})
\end{itemize}

\section{Historia y evolución}\label{sec:hist}
El concepto de \textit{detección de intrusiones} (IDS) fue introducido por Dorothy Denning en
1987~\cite{denning1987intrusion}. En su trabajo, Denning propuso un sistema de detección de intrusiones
basado en el análisis de los registros de auditoría del sistema. Este sistema se basaba en la idea
de que las actividades de los usuarios legítimos del sistema se pueden modelar y, por lo tanto,
se pueden detectar las actividades que no se ajustan a dicho modelo. Este trabajo fue el punto de
partida de la detección de anomalías.

Pese a que el trabajo de Denning fue la primera aplicación real en la historia de la detección de
anomalías, ya existían algunas técnicas que se pueden considerar como predecesoras de la minería
de excepciones. Por ejemplo, en \textit{Economic Control of Quality of
Manufactured Product}~\cite{shewhart1931economic}, se propusieron técnicas de clasificación de
conjuntos de datos gausianos en función de su desviación estándar. Este tipo de técnicas se pueden
considerar como predecesoras de las técnicas de detección de anomalías basadas en la distancia
y son estrategias rudimentarias aun utilizadas en la actualidad.

\noindent
\begin{minipage}{\linewidth}
	\centering
	\includegraphics[width=\textwidth]{gaussian.png}
	\captionof{figure}{Regla 68{-}95{-}99.7 para desv.~estándar en
		medias de distr.~gausianas~\cite{galarnyk2018explaining}}\label{fig:fig1}
\end{minipage}
\vspace{1\baselineskip}

La detección de intrusiones es un componente crítico de la detección de anomalías y ha ido evolucionando
significativamente con el paso del tiempo. En la actualidad, los sistemas de detección de intrusiones
se basan en técnicas de aprendizaje automático y de minería de datos, reflejando el progreso de la minería
de excepciones.

El término \textit{minería de anomalías} fue introducido por primera vez por \textit{Patcha y Park} en 2007
~\cite{patcha2007overview}.

\chapter{Técnicas y métodos de detección}
\section{Revisión de métodos}

\section{Enfoques}

\chapter{Preprocesamiento de datos}\label{chap:pre}
El preprocesamiento de datos es una fase crítica que implica la limpieza,
transformación y preparación de los datos, en este caso previo a la aplicación
de técnicas de detección de anomalías.

\nocite{herrera2004pre}

\subsection{¿Por qué es necesario?}
Los datos que se obtienen de las fuentes de información suelen estar incompletos,
contener errores o no ser adecuados para el problema que se quiere resolver.
Por ello, es necesario realizar una serie de operaciones sobre los mismos para
poder utilizarlos de forma efectiva.

Además, la preparación del conjunto puede resultar en una reducción de la cantidad
de datos a procesar, lo que puede suponer una mejora en la eficiencia del proceso
a nivel global, uno de los problemas vistos en el capítulo~\ref{sec:issues}.

\subsection{¿Cómo se hace?}
El preprocesamiento de datos se puede dividir en varias fases:
\section{Técnicas de preprocesamiento}

\section{Selección de características}

\chapter{Casos de uso}
\section{Casos reales}

\section{Ejemplos de uso}

\chapter{Desafíos y consideraciones éticas}
\section{Problemas comunes}\label{sec:issues}
A la hora de tratar de utilizar y evaluar métodos de detección de anomalías, existen una
serie de problemas comunes que se deben tener en cuenta:

\begin{itemize}[topsep=0pt]
	\item \textbf{Evaluación:} a la hora de evaluar la calidad de un clasificador, las medidas que se tomen
		pueden no ser adecuadas para el problema en cuestión. Por ejemplo, en el caso de los fraudes,
		la tasa de falsos positivos puede ser más importante que la tasa de falsos negativos.
	\item \textbf{Eficiencia y escalabilidad:} debido a la gran cantidad de datos que se manejan y los
		recursos computacionales que se necesitan para crear un clasificador ajustado a estos, la eficiencia
		es un factor a tener en cuenta.
	\item \textbf{Número de atributos:} dependiendo del conjunto de datos con el que se esté tratando, en
		ocasiones puede ser necesario \emph{más de un atributo} para \textbf{definir} una anomalía.
	\item \textbf{Localidad:} a la hora de detectar anomalías, es importante tener en cuenta tanto el
		\emph{conjunto} como el \emph{entorno} en el que se encuentra cada objeto. Como visto en clase,
		una persona puede ser considerada ``anormalmente'' alta para la población \textit{normal},
		pero no dentro del conjunto \textit{jugadores de baloncesto}.
	\item \textbf{Grado de anormalidad:} la propiedad de ser anormal no es binaria, sino que se le puede
		asignar un \emph{grado de anormalidad}, que normalmente define la distancia entre el objeto y el
		conjunto de datos normales. Este concepto es importante y es la base de muchas técnicas de detección
		no supervisadas o semi-supervisadas (Ver \nameref{chap:tecnicas}).
	\item \textbf{Múltiples anomalías simultáneas:} en ocasiones, un objeto puede ser considerado anómalo
		por múltiples razones. En relación con el punto anterior, el hecho de que dos objetos tengan el mismo
		\textit{grado} de anormalidad no implica que sean anómalos por los mismos motivos (\textit{atributos}).
\end{itemize}
\newpage{}
Estos son algunos de los posibles problemas que se pueden encontrar a la hora de desarrollar un sistema de
este tipo, pero existen muchos más:
\begin{itemize}
	\item Preprocesado incorrecto o insuficiente
	\item Conjuntos desbalanceados
	\item Sesgo (Ver \nameref{sec:etica})
	\item Interpretación de los resultados
	\item \ldots
\end{itemize}

\section{Aspectos éticos}\label{sec:etica}
Aunque la detección de anomalías sea una herramienta muy poderosa a la hora de identificar
fraudes, errores o problemas, también plantea desafíos éticos a tener en cuenta.

La primera gran categoría de dichos problemas reside en la obtención del conjunto en sí.
A la hora de recoger datos, es importante tener en cuenta la privacidad y confidencialidad
de las partes interesadas en los mismos, ya que puede contener información sensible. Además
de tener en cuenta la privacidad, es importante considerar la legalidad del tratamiento de
dichos datos. Como se ha visto en otras asignaturas de este curso, la toma de decisiones
automatizada sobre individuos o el tratamiento de datos personales se encuentra regulada
por la \textit{Ley Orgánica de Protección de Datos} (LOPD)~\cite{lopd}.

Otra categoría de problemas éticos reside en la calidad del conjunto. En ocasiones, los
conjuntos de datos pueden contener sesgos, ya sea por la forma en la que se han recogido
los datos o por la propia naturaleza de los mismos. Por ejemplo, en el caso de los fraudes,
es posible que los datos que se tengan sobre fraudes sean de un tipo concreto, por lo que
el clasificador no será capaz de detectar fraudes de otro tipo. Este problema se conoce
como \textit{sesgo de selección}. Otro ejemplo de sesgo es el \textit{sesgo de supervivencia},
que se produce cuando los datos que se tienen son de los objetos que han sobrevivido a un
proceso de selección, pero no de los que no lo han hecho.

Por último, es importante tener en cuenta las posibles consecuencias de la minería de
anomalías. En ocasiones, los resultados de la detección de anomalías pueden tener un
impacto negativo en las personas o entidades involucradas. De otra forma, se pueden
obtener resultados incorrectos o mal interpretados, lo que puede llevar a tomar decisiones
erróneas.

Como conclusión, es importante tener en cuenta que la minería de anomalías es una herramienta
muy potente pero con potenciales consecuencias negativas, en especial contra la privacidad
de individuos y otras partes interesadas. A parte de los problemas éticos, también se deben
tener en cuenta los problemas legales que se pueden derivar de su uso.

\chapter{Herramientas y tecnologías utilizadas}
Para el desarrollo de modelos de predicción de minería de excepciones, existen multitud de
herramientas y tecnologías, algunas de ellas de código abierto, que facilitan el proceso de
desarrollo. Este capítulo tiene como objetivo mencionar algunas de estas herramientas y
tecnologías.

\begin{itemize}
	\item \textbf{ELKI} es un toolkit de minería de datos de código abierto escrito en Java. Está
		diseñado para ser utilizado en investigación y en la industria, y es compatible con la mayoría
		de sistemas operativos. Cuenta además con índices de aceleración especiales.
	\item \textbf{PyOD} es una librería de Python de código abierto desarrollada específicamente
		para la detección de anomalías.~\cite{pyod}
	\item \textbf{scikit-learn} es una librería de Python de código abierto que recoge algoritmos
		de detección de anomalías no supervisados. (Ver \nameref{chap:tecnicas})
	\item \textbf{Wolfram Mathematica} es un software de matemáticas que incluye herramientas de
		minería de datos y de aprendizaje automático de conjuntos de datos de varios
		tipos.~\cite{wolfram}
\end{itemize}

Además de las herramientas mencionadas, las empresas que necesitan modelos y clasificadores más
complejos suelen recurrir a la creación de herramientas propias que se adapten a sus necesidades
y a los servicios y datos que ya tienen. Algunos ejemplos de estas empresas son las vistas en el
capítulo anterior, como Netflix, Uber, Pinterest o LinkedIn.~(Ver \nameref{sect:reales})

\chapter{Conclusiones}
\section{Aprendizaje y conclusiones}

\section{\LaTeX}
Este trabajo sirve como introducción personal a \LaTeX{} y a la escritura de documentos formales, preparándome así para el Trabajo Fin de Grado.
El código fuente de este documento está disponible en un repositorio de GitHub~\cite{source}.


%% Esto incluirá la bibliografía correctamente en nuestro trabajo
\newpage % En una nueva página
\addcontentsline{toc}{chapter}{Bibliografía} % Añade la referencia al índice de contenido

\bibliographystyle{ieeetr} % Define el estilo de la bibliografía
\bibliography{biblio} % Indica el archivo que contiene la colección de citas

\nocite{apuntes}
\nocite{leal2009aplicacion}
\nocite{perez2020uso}

\end{document}
