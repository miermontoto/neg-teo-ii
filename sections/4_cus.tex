\chapter{Casos de uso}\label{chap:cus}
\section{Ejemplos de uso}
La minería de excepciones se puede aplicar a un gran abanico de casos, en múltiples campos.
Algunos de los más frecuentes son los siguientes:
\begin{itemize}
	\item Detección del fraude
	\item Detección de intrusiones en sistemas informáticos
	\item Detección de caídas de servicios
	\item Salud pública y medicina
	\item Transmisión de datos en redes
	\item Seguridad militar y nacional
\end{itemize}

Todos estos casos de uso hacen uso principalmente de la minería de anomalías, aplicando las
técnicas vistas en el capítulo~\ref{chap:tecnicas} para detectar desviaciones que indiquen un
comportamiento anómalo.

\section{Casos reales}\label{sect:reales}
Como ya se ha mencionado antes (Ver \nameref{sec:hist}), la detección de anomalías comenzó a partir
de la detección de intrusiones de Dorothy Denning~\cite{denning1987intrusion}. Desde entonces, se
han desarrollado diferentes técnicas y usos para la detección de anomalía. En esta sección se
verán algunos ejemplos de casos reales expandiendo de los ejemplos de uso vistos en la sección
anterior.

\begin{itemize}
	\item En \textbf{1999}, científicos de Bell Atlantic desarrollaron un sistema de detección de
		anomalías para detectar fraudes en las llamadas telefónicas~\cite{fawcett1999activity}.
	\item En \textbf{2005}, investigadores de las universidades del estado de Oregon y Carnegie Mellon
		propusieron algoritmos que aplicaban la detección de anomalías a la detección temprana de
		brotes de enfermedades~\cite{wong2005wsare}.
	\item En \textbf{2015}, Netflix publicó un artículo en el que explicaba cómo utilizaban la
		detección de anomalías genéricas, con enfoque en conjuntos de alta dimensionalidad y tratando
		de minimizar los falsos positivos (\emph{RAD})~\cite{netflix2015rad}.
	\item También en ese año, Uber publicó un artículo explicando el uso de su herramienta de minería
		de excepciones \emph{Argos}, que utilizan para detectar caídas de sus servicios y encontrar su
		causa.~\cite{uber2015argos}
	\item En \textbf{2016}, investigadores de la IEEE publicaron un sistema de detección de anomalías
		para las transmisiones entre WSNs (\textit{Wireless Sensor Networks}) mediante el análisis de
		vecinos más cercanos (una de las técnicas vistas~\nameref{chap:tecnicas}).~\cite{ieee2016wsn}
	\item En \textbf{2019}, Pinterest publicó un artículo en el que explicaba cómo utilizaban la
		detección de anomalías de manera escalable para analizar las métricas de sus
		servicios.~\cite{pinterest2019anomaly}
	\item También en ese año, la popular plataforma LinkedIn publicó un artículo del mismo estilo,
		explicando la integración de su herramienta \textit{ThirdEye} en sus servicios de
		negocio.~\cite{linkedin2019thirdeye}
\end{itemize}
