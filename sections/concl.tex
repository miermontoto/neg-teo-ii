\chapter{Conclusiones}
\section{Anomalías}
La minería de anomalías es un campo muy interesante que tiene aplicaciones en muchos
campos diferentes. Este trabajo no tiene un enfoque específico en la inteligencia de
negocio, sino que abarca un espectro más amplio de aplicaciones. En mi opinión, es
una parte muy importante de la minería de datos que no se suele tratar en profundidad,
además de tener un gran potencial en el mundo real.

Es evidente que una gran cantidad de empresas de nuestro sector utilizan técnicas
propias de detección de anomalías con sistemas muy robustos y complejos, por lo que
es un tema a tener en cuenta en el futuro.

\section{Aprendizaje}
A lo largo de este trabajo, he podido profundizar lo visto en clase sobre la minería de
anomalías, el tema más interesante (en mi opinión) de todos los presentados para este
trabajo. He aprendido sobre las diferentes técnicas, un pequeño repaso sobre el
preprocesamiento y, mi parte favorita, he podido investigar las aplicaciones reales
para los casos de uso vistos.

El tiempo me limita, igual que en mi trabajo anterior sobre \textit{Dashboards}, a
poder profundizar más, pero creo que es un tema muy interesante que no voy a olvidar
tan fácilmente.

\section{\LaTeX}
Este trabajo sirve como mi introducción personal a \LaTeX{} y a la escritura de documentos formales, preparándome así para el Trabajo Fin de Grado.
El código fuente de este documento está disponible en mi repositorio de GitHub~\cite{source}.

Además de el mero uso del lenguaje de marcado, he aprendido a citar y referenciar de forma correcta,
he desarrollado un entorno propio de plantillas que me permite crear documentos formales de forma rápida y
sencilla y he aprovechado mi aprendizaje de todos estos años para mejorar la calidad de mis trabajos,
cosa que espero que se vea reflejada en la calidad de trabajos futuros.
