\chapter{Introducción}
\section{Definición}
En esencia, la minería de anomalías es una rama de la minería de datos que se enfoca en
la detección de patrones anómalos o inusuales en los datos.

Estas anomalías representan desviaciones significativas de los comportamientos esperados
o de las tendencias habituales en un entorno comercial.

\section{Objetivos y aplicaciones en el mundo real}
El objetivo principal radica en identificar estos eventos ``atípicos'' que podrían llegar
a indicar problemas o posibles oportunidades, si lo observamos desde el punto de vista de
la \subject.

En el mundo real, la minería de anomalías se puede aplicar (y se aplica) a la hora de
tomar decisiones empresariales, como por ejemplo al detectar fraudes, revelar áreas de
mejora, detectar problemas operativos\ldots

\section{Causas de las anomalías}
Las anomalías pueden ser causadas por una amplia variedad de razones, pero existen algunas
causas comunes que pueden ser identificadas:

\begin{itemize}
	\item \textbf{Datos de clases diferentes:} un objeto puede ser diferente del resto por
		pertencer a una clase diferente. En el ejemplo de los fraudes, los objetos ``outliers''
		dentro de un conjunto de datos pueden ser considerados como inusuales o como potenciales
		fraudes.
	\item \textbf{Variación natural:} cuando los datos se pueden modelar estadísticamente
		mediante una distribución, los objetos que se encuentran en las colas de la distribución
		se pueden llegar a considerar como anomalías.
	\item \textbf{Errores en la medición:} todos los datos que se recogen están sometidos a una
		componente humana y, por lo tanto, a errores. Dichos errores pueden llegar a ser
		considerados anomalías, por lo que hay que lidiar con ellos en la fase de preprocesamiento.
\end{itemize}
