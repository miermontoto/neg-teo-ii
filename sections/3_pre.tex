\chapter{Preprocesamiento de datos}\label{chap:pre}
El preprocesamiento de datos es una fase crítica que implica la limpieza,
transformación y preparación de los datos, en este caso previo a la aplicación
de técnicas de detección de anomalías.

\nocite{herrera2004pre}

\subsection{¿Por qué es necesario?}
Los datos que se obtienen de las fuentes de información suelen estar incompletos,
contener errores o no ser adecuados para el problema que se quiere resolver.
Por ello, es necesario realizar una serie de operaciones sobre los mismos para
poder utilizarlos de forma efectiva.

Además, la preparación del conjunto puede resultar en una reducción de la cantidad
de datos a procesar, lo que puede suponer una mejora en la eficiencia del proceso
a nivel global, uno de los problemas vistos en el capítulo~\ref{sec:issues}.

\subsection{¿Cómo se hace?}
El preprocesamiento de datos se puede dividir en varias fases:
\section{Técnicas de preprocesamiento}

\section{Selección de características}
