\chapter{Desafíos y consideraciones éticas}
\section{Problemas comunes}\label{sec:issues}
A la hora de tratar de utilizar y evaluar métodos de detección de anomalías, existen una
serie de problemas comunes que se deben tener en cuenta:

\begin{itemize}[topsep=0pt]
	\item \textbf{Evaluación:} a la hora de evaluar la calidad de un clasificador, las medidas que se tomen
		pueden no ser adecuadas para el problema en cuestión. Por ejemplo, en el caso de los fraudes,
		la tasa de falsos positivos puede ser más importante que la tasa de falsos negativos.
	\item \textbf{Eficiencia y escalabilidad:} debido a la gran cantidad de datos que se manejan y los
		recursos computacionales que se necesitan para crear un clasificador ajustado a estos, la eficiencia
		es un factor a tener en cuenta.
	\item \textbf{Número de atributos:} dependiendo del conjunto de datos con el que se esté tratando, en
		ocasiones puede ser necesario \emph{más de un atributo} para \textbf{definir} una anomalía.
	\item \textbf{Localidad:} a la hora de detectar anomalías, es importante tener en cuenta tanto el
		\emph{conjunto} como el \emph{entorno} en el que se encuentra cada objeto. Como visto en clase,
		una persona puede ser considerada ``anormalmente'' alta para la población \textit{normal},
		pero no dentro del conjunto \textit{jugadores de baloncesto}.
	\item \textbf{Grado de anormalidad:} la propiedad de ser anormal no es binaria, sino que se le puede
		asignar un \emph{grado de anormalidad}, que normalmente define la distancia entre el objeto y el
		conjunto de datos normales. Este concepto es importante y es la base de muchas técnicas de detección
		no supervisadas o semi-supervisadas (Ver \nameref{chap:tecnicas}).
		\begin{itemize}
			\item \textbf{Local Outlier Factor (\emph{LOF}):} el LOF es un tipo de grado de anormalidad
				calculado a partir de la densidad \textbf{local} a partir de la distancia entre los puntos
				más cercanos (vecinos). Este método es capaz de detectar anomalías locales, pero no globales,
				en relación con la idea de \textit{localidad} mencionada anteriormente.
		\end{itemize}
	\item \textbf{Múltiples anomalías simultáneas:} en ocasiones, un objeto puede ser considerado anómalo
		por múltiples razones. En relación con el punto anterior, el hecho de que dos objetos tengan el mismo
		\textit{grado} de anormalidad no implica que sean anómalos por los mismos motivos (\textit{atributos}).
\end{itemize}
\newpage{}
Estos son algunos de los posibles problemas que se pueden encontrar a la hora de desarrollar un sistema de
este tipo, pero existen muchos más:
\begin{itemize}
	\item Preprocesado incorrecto o insuficiente
	\item Conjuntos desbalanceados
	\item Sesgo (Ver \nameref{sec:etica})
	\item Interpretación de los resultados
	\item \ldots
\end{itemize}

\section{Aspectos éticos}\label{sec:etica}
Aunque la detección de anomalías sea una herramienta muy poderosa a la hora de identificar
fraudes, errores o problemas, también plantea desafíos éticos a tener en cuenta.

La primera gran categoría de dichos problemas reside en la obtención del conjunto en sí.
A la hora de recoger datos, es importante tener en cuenta la privacidad y confidencialidad
de las partes interesadas en los mismos, ya que puede contener información sensible. Además
de tener en cuenta la privacidad, es importante considerar la legalidad del tratamiento de
dichos datos. Como se ha visto en otras asignaturas de este curso, la toma de decisiones
automatizada sobre individuos o el tratamiento de datos personales se encuentra regulada
por la \textit{Ley Orgánica de Protección de Datos} (LOPD)~\cite{lopd}.

Otra categoría de problemas éticos reside en la calidad del conjunto. En ocasiones, los
conjuntos de datos pueden contener sesgos, ya sea por la forma en la que se han recogido
los datos o por la propia naturaleza de los mismos. Por ejemplo, en el caso de los fraudes,
es posible que los datos que se tengan sobre fraudes sean de un tipo concreto, por lo que
el clasificador no será capaz de detectar fraudes de otro tipo. Este problema se conoce
como \textit{sesgo de selección}. Otro ejemplo de sesgo es el \textit{sesgo de supervivencia},
que se produce cuando los datos que se tienen son de los objetos que han sobrevivido a un
proceso de selección, pero no de los que no lo han hecho.

Por último, es importante tener en cuenta las posibles consecuencias de la minería de
anomalías. En ocasiones, los resultados de la detección de anomalías pueden tener un
impacto negativo en las personas o entidades involucradas. De otra forma, se pueden
obtener resultados incorrectos o mal interpretados, lo que puede llevar a tomar decisiones
erróneas.

En conclusión, la minería de anomalías es una herramienta muy potente pero con potenciales
consecuencias negativas, en especial contra la privacidad de individuos. A parte de los
problemas éticos, también se deben tener en cuenta los problemas legales que se pueden
derivar de su uso incorrecto.
